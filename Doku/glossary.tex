\newacronym{rtos}{RTOS}{Real Time Operation System}
\newacronym{os}{OS}{Operating System}
\newacronym{bu}{BU}{Business Unit}
\newacronym{cpu}{CPU}{Central Processing Unit}
\newacronym{fubar}{FUBAR}{Fucked Up Beyond All Repair}
\newacronym{isr}{ISR}{Interrupt Service Routine}
\newacronym{rd}{R \& D}{Research and Development}
\newacronym{pm}{PM}{Projektmanagement}
\newacronym{ssu}{SSU}{Service Sales Units}
\newacronym{cu}{CU}{Central Unit}
\newacronym{gbc}{GBC}{Global Business Center}
\newacronym{mpu}{MPU}{Memory Protection Unit}
\newacronym{dma}{DMA}{Direct Memory Access}
\newacronym{pwm}{PWM}{Pulsweitenmodulation}
\newacronym{gui}{GUI}{engl. "Graphical User Interface", grafische Benutzeroberfläche}
\newacronym{mqtt}{MQTT}{engl. "Message Queuing Telemetry Transport"}
\newacronym{http}{HTTP}{Hyper Text Transfer Protocoll}
\newacronym{reluktanz}{Reluktanz}{Magnetischer Widerstand}
\newacronym{koinzidenzstellung}{Koinzidenzstellung}{Stellung mit magnetischer Vorzugsrichtung}
\newacronym{haltemoment}{Haltemoment}{Maximales Drehmoment, mit dem ein stromloser Schrittmotor ohne Verdrehung belastet werden kann}
\newacronym{vrMotor}{VR-Motor}{variable reluctance motor (Reluktanz)}
\newacronym{synchronmaschine}{Synchronmaschine}{Elektrischer Motor, bei dem der Rotor synchron mit dem Drehfeld des Stators läuft.}
\newacronym{cutoffvoltage}{Cut-Off Voltage}{Minimale Zellspannung, bei der die Energiezufuhr durch eine Li-Ionen Batterie abgeschaltet werden sollte}
\newacronym{gpio}{GPIO}{engl. "General Purpose Input/Output", programmierbarer Ein- und Ausgang	für allgemeine Zwecke}
\newacronym{reed}{Reed}{Reedschalter oder -kontakte sind in einem Glasrohr eingeschmolzene Kontakte aus einem ferromagnetischen Material. Durch Magnetismus wird der Kontakt geschlossen.}
\newacronym{lagekopplung}{Lagekopplung}{Schätzung der Lage eines Fahrzeugs anhand der Daten des Antriebs- systems (bspw. Odometriedaten)}
\newacronym{uart}{UART}{engl. "Universal Asynchronous Receiver/Transmitter"; Serielle Komunikation zwischen zwei Controllern; Im Falle des Arduinos über die Pins Rx/ Tx als TTL-Logikpegel}
\newacronym{rx}{RX}{Receive}
\newacronym{tx}{TX}{Transmit}
\newacronym{mvc}{MVC}{Model View Controller}

\newglossaryentry{report}
{
    name=Bericht,
    description={Eine strukturierte Darstellung von Inhalten aus Datenbeständen}
}

\newglossaryentry{gnd}
{
    name=GND,
    description={Ground; Der -Pol eines Stromkreises}
}
