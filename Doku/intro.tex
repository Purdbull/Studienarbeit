
\chapter{Einleitung}

%\section{Vorwort}
\section{Kurzzusammenfassung}
2018 entwickelten Moritz Knapp und sein Vater Johannes Knapp eine Gartenhochbahn, welche vom Küchenfenster bis zur Terrasse fährt. 
Sinn und Zweck dieser Bahn ist der Transport von Essen und Getränken wie Frühstück oder Café. Die Bedienung der Hochbahn erfolgte bis dato analog. Während das Grundkonzept erhalten bleibt, wird mit dieser Arbeit der Aufbau zu einem vollautomatischem Transportsystem erweitert.

\section{Problemstellung und Zielsetzung}
    
Bislang war die Hochbahn in ihrer Funktionsweise stark eingeschränkt. Des Weiteren muss das Antriebskonzept in einigen Punkten überarbeitet werden. Ziel dieser Arbeit ist die Entwicklung einer digitalen Schnittstelle zum Nutzer. Dabei soll die Bahn eine hohe Zuverlässigkeit und Verfügbarkeit aufweisen. 
\newpage

\section{Aufbau der Arbeit}
In diesem schriftlichen Teil der Studienarbeit wird mit \autoref{cha:grundlagen} zunächst auf die Grundlagen eingegangen. Dazu zählt die Theorie zur \textit{Abstandsmessung mit Ultraschall}, dem \textit{Schrittmotor} sowie der Kommunikation mit \textit{\acrshort{mqtt}} und \textit{\acrshort{uart}}. 
Mit \autoref{cha:anforderungsanalyse} werden die Anforderungen an das Transportsystem Gartenhochbahn gestellt. Nach der Analyse in \textit{Ausgangslage} werden die Forderungen an das neue Transportsystem gestellt. Die Vorgehensweise des Projektes wird in \autoref{cha:vorgehensweise} beschrieben. Aufgrund der gestellten Anforderungen erfolgt mit \autoref{cha:konzeptionierung} die Konzeptionierung. \autoref{cha:umsetzung} beschäftigt sich mit der Umsetzung dieser Konzepte. 
 




