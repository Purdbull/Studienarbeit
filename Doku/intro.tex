\chapter{Einleitung}

%\section{Vorwort}
\section{Kurzzusammenfassung}
2018 entwickelten Moritz Knapp und sein Vater Johannes Knapp eine Gartenhochbahn, welche vom Küchenfenster bis zur Terrasse fährt. 
Sinn und Zweck dieser Bahn ist der Transport von Essen und Getränken wie Frühstück oder Cafe. Die Bedienung der Hochbahn erfolgte bis dato nur analog. Während das Grundkonzept beibehalten bleibt, wird mit dieser Arbeit der Aufbau zu einem vollautomatischem Transportsystem erweitert. Dazu zählt die digitale Erfassung von Bestellungen sowie die automatisierte Lieferung der Speisen  und Getränke. 

\section{Problemstellung und Zielsetzung}
    
Bislang war die Hochbahn in ihrer Funktionsweise stark eingeschränkt. Des Weiteren muss das Antriebskonzept in einigen Punkten überarbeitet werden. Ziel dieser Arbeit ist die Entwicklung eines digitalen Bestellprozesses, bei dem der Liefervorgang automatisch abläuft. Dabei soll die Bahn eine hohe Zuverlässigkeit und Verfügbarkeit aufweisen. 
\newpage

\section{Aufbau der Arbeit}
In diesem schriftliche Teil der Studienarbeit wird mit \autoref{cha:grundlagen} zunächst auf die Grundlagen eingegangen. Dazu zählt die Theorie zur \textit{Abstandsmessung mit Ultraschall}, dem \textit{Schrittmotr} sowie der Komunikation mit \textit{\acrshort{mqtt}}. 
Mit \autoref{cha:anforderungsanalyse} werden die Anforderungen an das Transportsystem Gartenhochbahn gestellt. Nach der Analyse der \textit{Ausgangslage} werden die Forderungen an das neue Transportsystem gestellt. Die Vorgehensweise des Projektes wird in \autoref{cha:vorgehensweise} beschrieben. Aufgrund der gestellten Anforderungen erfolgt mit \autoref{cha:konzeptionierung} die Konzeptioinierung. \autoref{cha:umsetzung} beschäftigt sich mit der Umsetzung der Konzepte. 

Der erste Teil der schriftlichen Ausarbeitung beschränkt sich überwiegend auf die Konzeptionierung der mechatronischen Komponenten. 