\chapter{Anforderungsanalyse}
Der bisherige Aufbau der Gartenhochbahn soll verbessert und in einigen Bereichen erneuert werden. 
In diesem Kapitel wird mit Abschnitt 3.1.%referenzieren!
 zunächst das bisherige System beschrieben. 
Mit Abschnitt 3.2. folgen die resultierenden Anforderungen unterteilt in die jeweiligen Bereiche.  

\section{Ausgangslage}
Die Gartenhochbahn stellt ein Transportsystem dar. Hauptbestandteil des Systems ist eine Gondel, die sich über Laufräder 
auf einer Schiene fortbewegt. 

Der Antrieb der Gondel erfolgt durch einem Gleichstrommotor, der über PWM (Pulsweitenmodulation) angesteuert wird. Für die Bereitstellung der Energie ist eine Batterie in der Gondel installiert. Diese wird in der Parkstellung am Ende der Schiene aufgeladen. Als Abschaltung des Motors an den hinteren Endlagen sind Tastsensoren verbaut. Ein Read-Kontakt in der Gondel erkennt außerdem Magnete entlang der Strecke. Dadurch kann bereits vor Erreichen der Endlage die Geschwindigkeit reduziert werden. Für die Signalverarbeitung und Steuerung des Motors ist ein Arduino Nano im Einsatz. \\

Die Kraftübertragung des Motors auf das treibende Laufrad erfolgt bislang über Reibung. Der Aufbau weist einen hohen Verschleiß auf und ist nicht ausreichend zuverlässig. 

Durch die Sensorik ist eine Abschaltung bei Erreichen der Endlage möglich. Somit können einzelne Fahrzyklen der Hochbahn automatisiert durchgführt werden. 



\section{Anforderungen}
\subsection{Antrieb}
Insgesamt soll der Antrieb eine hohe Verfügbarkeit aufweisen und verschleißarm sein. 
Die Kraftübertragung soll mit einer passenden Übersetzung erfolgen. Durch den elektrischen Motor soll eine Regelung der Geschwindigkeit und Drehrichtung möglich sein. Wünschenswert wäre außerdem eine Rückmeldung der Umdrehungen. 
Das Prinzpi der Energieversorgung über eine Batterie soll beibehalten werden. Dafür soll der Akkumulator bei möglichst geringem Platzbedarf und Masse eine ausreichende Kapazität besitzen. 

\subsection{Bestellsystem}

Ideen: 

\begin{itemize}
	\item [-] Speisekarte auf dem Webserver. --> Auskunft über Bestand
	\item [-] Benutzerfreundliche GUI
	\item [-] Slebst organisierend, dh. Priorisierung oder Queu der Aufträge? 
	\item [-] Lieferungszustand einsehen können (mit Positionserkennung)
	\item [-] Bestelllübersicht mit Rechnung? 
	\item [-] Zugriff von beliebigen Endgegnern (Endgeräten :D --> loool)
\end{itemize}

\subsection{Sensorik}
Die erforderliche Sensorik kann in zwei Bereiche gegliedert werden: 

\begin{itemize}
	\item [a)] Positionserkennung 
	\item [b)] Sicherheitsabschaltung 
	
\end{itemize}

Die Positionserkennung soll entlang der Strecke erfolgen. Die Information soll der Steuerung zur Verfügung gestellt werden. Dabei sollend die Positionswerte so exakt wie nötig ermittelt werden.

Die Sicherheitsabschaltung soll Hindernisse im Fahrweg der Gondel erkennen und dadurch einen rechtzeitigen Stillstand gewährleisten. Dafür muss die Sensorik den kritischen Bereich ausreichend prüfen. 





\chapter{Konzept}
\section{Mechanik}
\section{Hardware}
\subsection{Platine}
\subsection{Stromversorgung}
\subsection{Positionserkennung}

\section{Software}
\subsection{Server}
\subsection{Client}
\subsection{Web-Applikation}
\subsection{Steuerung und Positionserkennung}

\chapter{Umsetzung}

\lipsum